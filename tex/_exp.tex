\appendix
\pagenumbering{Roman}

\chapter{Cizí výrazy}
\begin{itemize}
\item \textbf{AAA (Authentication, Authorization, Accounting protocol)} - Sada bezpečnostní protokolů sloužící k autentizaci a autorizaci.
\item \textbf{Bug} - Chyba v programu
\item \textbf{CaaS (Container as a Service)} - CaaS je služba, která pomáhá vývojářům a operátorům vytvářet a ovládat kontejnery. 
\item \textbf{CLI (Command Line Interface)} - Textové uživatelské rozhraní
\item \textbf{Cloud-init} - Program, který se používá u cloudových serverů, slouží ke spuštění nadefinovaných akcích hned po spuštění instance. Bývá používán k inicializačnímu nastavení instance. 
\item \textbf{Cluster} - Seskupení skupiny serverů, které spolu spolupracují a chovají se jako jeden server.
\item \textbf{cgroups, namespaces} - Zdroje zabudované v Linuxovém jádře, pomocí kterých je zajištěna izolace kontejnerů. Cgroups slouží k izolaci CPU, paměti a diskových zdrojů, namespaces pak slouží k izolaci procesů. 
\item \textbf{Commit} - Základní stavební jednotka verzovací systému git. Commit udržuje záznamy změn v zdrojových kódech aplikace. 
\item \textbf{CRM (Customer relationship management)} - Software sloužící ke komunikaci mezi firmou a zákazníky např. ServiceNow, SalesForce.
\item \textbf{Development} - Vývojářský tým
\item \textbf{Downstream repozitář} - Downstream je kopie upstream repozitáře, který obsahuje změny a nové vlastnosti, které nejsou v upstreamu dostupné. Downstream tvoří společnosti, které mají potřebu do projektu dodat novou funkcionalitu či opravnou záplatu a nechtějí čekat na komunitní procesy. Do downstream větví mohou přispívat pouze kontributoři, kteří mají do repozitáře přístup.
\item \textbf{Git} - Git je nástroj na správu verzí zdrojových kódů aplikace.
\item \textbf{GitHub} - Github je služba pro sdílení zdrojových kódů aplikací vezovaných gitem. Většina open source projektů je hostována na serveru GitHub.com 
\item \textbf{Hypervisor} - Server, který slouží jako host pro virtuální stroje.
\item \textbf{IDM (Identity Management)} - IDM je služba, zajišťující správu identity. Jedná se o služby, které spravují uživatele a jejich oprávnění např. FreeIPA, AD.
\item \textbf{Linuxové jádro (kernel)} - Monolitické jádro, které ovládá a spravuje hardwarové zdroje na linuxových distribucích.  
\item \textbf{Open source} - Otevřený zdrojový kód, tento kód je dostupný na internetu pod otevřenou licencí např. Apache2, GNU GPL. 
\item \textbf{Operations} - Tým zodpovědný za provoz infrastruktury a správu aplikací.
\item \textbf{PagerDuty} - Platforma sloužící ke správě notifikací z infrastruktury.
\item \textbf{Private cloud} - Private cloud je určený jen pro určitou skupinu uživatelů a dostupný pouze z jejich sítě. Je hostován a provozován na vlastním hardwaru. Nejpoužívanější technologie pro private cloud jsou Microsoft Azure Pack a OpenStack.
\item \textbf{Public cloud} - Public cloud neboli veřejný cloud je poskytovatel, který nabízí pronájem virtuálních zdrojů prostřednictvím internetu. Největšími poskytovateli public cloudu jsou firmy Amazon (AWS), Google (GCE) a Microsoft (Azure).
\item \textbf{QA (Quality assurance)} - Sada procesů sloužící k testování a udržování aplikace.
\item \textbf{REST API (Representational State Transfer)} - REST je architektura komunikačního rozhraní, pomocí kterého lze pomocí HTTP volání provádět CRUD akce. 
\item \textbf{Runtime} - Runtime je sada systémových prostředků a knihoven, které slouží pro spouštění aplikací, např. JRE (Java Runtime Environment)
\item \textbf{Slack} - Komunikační nástroj sloužící k posílání zpráv, volání a sdílení souborů.
\item \textbf{Upstream repozitář} - Upstream repozitář, obsahuje zdrojové kódy projektu, které jsou veřejně přístupné. Vývoj v upstream repozitáři je řízen buď majitelem repozitáře nebo komunitou okolo projektu.
\item \textbf{VM (Virtual Machine)} - Virtuální stroj je produkt plné virtualizace. Pomocí emulovaného hardwaru je vytvořen virtualizovaný hardware, na kterém je pak možné spustit operační systém. 
\item \textbf{Vendor} - Společnost, která prodává či poskytuje podporu pro danou technologii.
\end{itemize}