\chapter{Závěr}
Kontejnerizace je dnes jedno z velmi aktuálních témat, především pak ve spojení s cloudovými technologiemi. Je to hlavně z důvodu konceptu zapouzdření logiky do funkčních celků, které poskytují vývojářům a systémovým administrátorům větší flexibilitu. Kontejnerizované služby lze zároveň díky své izolaci velmi jednoduše a rychle nasazovat do produkce, což zkracuje dobu, během které jsou vývojáři schopni doručovat zákazníkům nové verze aplikací.

 Cílem práce  bylo ověření, zda je možné převést tradiční aplikaci postavenou nad plnou virtualizací do kontejnerového prostředí a otestovat, zda budou vyřešeny problémy, se kterými se potýkala předchozí verze aplikace. V práci byl představen koncept kontejnerové virtualizace, byly zde porovnány jednotlivé architektury aplikací a zároveň představeny technologie, které jsou s kontejnerizací spojeny. V další části diplomové práce byla analyzována aplikace a byly stanoveny funkční požadavky, ze kterých vznikla nová architektura aplikace. Ta byla prakticky naimplementována a otestována. Z výsledku testování lze usoudit, že tradiční aplikaci do kontejnerového prostředí je možné zmigrovat. Aby migrace do kontejnerového prostředí byla úspěšná, je třeba připravit aplikaci, kde je nutno udělat spoustu změn.Vzhledem k množství a složitosti úprav aplikace pak ne vždy bývá tento postup migrace do kontejnerového prostředí efektivní. Pro migraci je pak vhodné vybrat správný nástroj, který přesunutí do kontejnerového prostředí zjednoduší a tím ulehčí práci nejen vývojářům ale i operátorům. 
 
Jako vhodný předmět pro další výzkum v rámci kontejnerizace je možné označit testování kontejnerizované aplikace na škále. Pomocí, kterého by se dalo zjistit, jak se chová v produkčním prostředí. Popřípadě se zamyslet nad tématem migrace více obecněji, zamyslet se nad univerzálním frameworkem, který by měl za cíl zjednodušit migraci tradičních aplikací do kontejnerového prostředí. Pomocí tohoto frameworku by pak mohl být značně ulehčen čas administrátorům a vývojářům, kteří musí aplikaci pro kontejnery optimalizovat.  
