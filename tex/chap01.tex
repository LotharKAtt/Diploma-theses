\chapter{Úvod}
\pagenumbering{arabic}
\setcounter{page}{1}
V dnešní době mobilních technologií a cloudových aplikací je kladen velký důraz na rychlý vývoj a doručování aplikací koncovým uživatelům. Tento trend můžeme sledovat především u dynamicky se rozvíjejících firem a startupů, které jsou schopny vydávat i několik verzí svých aplikací za měsíc. Vývoj aplikací se za posledních deset let dramaticky zrychlil. K vytvoření aplikace už není potřeba složitě designovat infrastrukturu, datacentra či ladit a optimalizovat databáze. Většina populárních technologií je vydávána jako open source a je částečně dostupná pro vývojáře zdarma. Infrastruktura v podobě fyzických serverů je často nahrazována cloudem, který je možno pronajmout a provozovat na něm téměř vše od databází až po sítě. Uživatel v cloudu platí pouze za využité zdroje, to znamená, že ušetří finanční prostředky spojené s provozem vlastního hardwaru. Využívání cloudu otevírá vývojářům nové možnosti pro řešení problému. Díky dostupnosti cloudových zdrojů se mohou vývojáři zaměřovat především na psaní kódu, místo řešení problému s infrastrukturou.

Naopak na druhé straně jsou ve firmách provozovány aplikace označované jako legacy. Tyto aplikace jsou součástí velkých systémů, které zajišťují kritickou funkcionalitu. Jedná se například o velké bankovní, armádní či informační systémy. Obrovským problémem těchto aplikací je jejich architektura. Tyto systémy byly navrhovány před několika desítkami let, proto je pro jejich správný běh nutný zastaralý operační systém a hardware. Aplikace bývají napsány v již nepodporovaných verzích programovacích jazyků a jejich zdrojové kódy bývají velmi složité na udržování. Jedná se o komplexní systémy, které byly upravovány spoustou rozdílných vývojářů. Dalším důležitým faktorem je dokumentace, která také nemusí být jednotná pro stejné verze aplikace. Problematické bývá testování aktuální verze a change management, protože tyto komplexní systémy bývají často napojeny na další systémy. Pak může i jedna malá změna v aplikaci změnit výsledné chování systému jako celku. Pro tvorbu těchto systémů byly používány neagilní vývojové metodiky, které neumožňovaly dostatečnou flexibilitu a komunikaci během vývoje.

Hlavní motivací proč migrovat tyto aplikace na novější technologie a architektury je jejich udržovatelnost. Jak již bylo zmíněno výše, tyto legacy aplikace převážně zajišťují chod důležitých systémů pro korporace a velké organizace. Čím je technologie starší, tím je nutné investovat více finančních prostředků k jejímu udržení, což se projeví i na zvýšených nákladech na dodatečné rozšíření aplikace. Pro udržování těchto aplikací jsou pak organizovány specializované týmy, které se zabývají pouze udržováním běhu aplikace a nasazováním nových či opravených verzí. Tyto verze bývají ovšem nasazovány velmi pomalu, především kvůli složitosti a náročnosti provést aktualizaci bez výpadků. Problém s pomalým nasazováním nových verzí vznikl v minulosti, kdy se světy vývojářů aplikací a systémových administrátorů začaly od sebe víc a více vzdalovat. Každý specializovaný tým se zabýval pouze svou danou částí, například programátoři končili svoji práci vytvořením instalátoru a otestováním aplikace v testovacím prostředí. Vývojáři se dále nezabývali problémem, jak funguje nasazená aplikace na produkčních serverech s reálnými daty. Naopak druhá strana, kterou zastávali systémoví inženýři, byla k novým verzím skeptická. Důvodem byla především komplexita a nové problémy (bugy), se kterými přicházely nové verze, které byly nasazovány. Problémy poté měli administrátoři především při údržbě aplikací a infrastruktury, protože jejich týmy byly odpovědné za funkčnost aplikací na produkčních serverech. Tento proces vedl k celkově obtížnému udržování aplikace. Proto se zhruba v letech 2006–2007 začaly v IT komunitě objevovat hlasy a argumenty poukazující na skutečnost, že současné řešení není optimální a že je potřeba vymyslet nový způsob propojení práce systémových administrátorů a vývojářů aplikací.

DevOps je přístup k vývoji a správě softwaru. Vznikl spojením dvou anglických slov: development (vývoj) a operation (správa). Tento přístup je založen na komunikaci mezi vývojáři a administrátory, kteří aplikaci v produkci spravovali. Celý tento přístup je založen na zrychlení vývoje aplikace a na snížení času, který je nutný k doručení nové verze aplikace či aktualizace ke koncovým zákazníkům. Vznik tohoto přístupu je datován kolem roku 2008 a je spjat s rozmachem cloudu a přístupem jako služba (as a service). Na rozdíl od agilního vývoje DevOps nemá žádný oficiální manifest nebo definici, každá firma či tým si ho vykládá po svém. DevOps přístup bourá komunikační bariéru mezi vývojáři softwaru a lidmi, kteří se o daný běh softwaru starají. Jde především o zodpovědnost týkající se provozu daného softwaru aplikace, už není možno označit, že za chyby v aplikaci může pouze jedna strana, veškeré problémy by měly být řešeny formou spolupráce a dialogu mezi týmy. Pomocí tohoto přístupu lze mnohem efektivněji zaručit rychlé řešení problémů na produkci.

Cílem této diplomové je zanalyzovat a přesunout vybranou legacy aplikaci do kontejnerového řešení. Výsledek práce spočívá ve vytvoření funkčních požadavků na kontejnerovou aplikaci, navržení architektury a vytvoření funkčního prototypu dané aplikace v kontejnerovém prostředí. V druhé kapitole je představen koncept kontejnerizace a základní technologie spojené s tímto typem virtualizace. Třetí kapitola se zabývá pojmem legacy aplikace a jsou zde popsány problémy monolitického přístupu, ten je poté porovnán s kontejnerovým řešením. Ve čtvrté kapitole je představena aplikace, která bude převedena do kontejnerového řešení. V páté kapitole jsou uvedeny funkční požadavky na kontejnerovou aplikaci a její architekturu. Zároveň se tato kapitola zabývá implementací aplikace do kontejnerového prostředí. V rámci šesté kapitoly je provedeno testování a porovnání aplikace na jednotlivých architekturách. V závěru diplomové práce je provedeno závěrečné shrnutí.

V práci se vyskytuje mnoho anglických výrazů a termínů. Jedná se zejména o slova, která nemají český ekvivalent, nebo jejich překlad nedává v kontextu smysl. Proto tato slova nebudou přeložena a budou ponechána v původním znění tak, jak se běžně v praxi používají. Komplikovanější termíny a zkratky budou vysvětleny v slovníku cizích výrazů, který je přiložen v závěru práce.